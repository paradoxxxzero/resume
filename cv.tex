\documentclass[a4paper,10pt]{article}

\usepackage[sfdefault,oldstyle]{roboto}
\usepackage[utf8x]{inputenc}
\usepackage[T1]{fontenc}
\usepackage[english]{babel}

\usepackage[big]{layaureo}
\usepackage{supertabular}
\usepackage{titlesec}
\usepackage{hyperref}
\usepackage{xcolor}

\definecolor{softBlack}{cmyk}{0, 0, 0, 0.9}
\color{softBlack}

\hypersetup{urlcolor=blue, colorlinks=true}

\titleformat{\section}{\Large\scshape\raggedright}{}{0em}{}[\titlerule]
\titlespacing{\section}{0pt}{3pt}{3pt}

\newif\iflong
% \longtrue

\begin{document}
  \pagestyle{empty}
  \par{\centering
    {\Huge Florian \textsc{Mounier}
    }\bigskip\par}

  \section{Informations personnelles}
  \begin{tabular}{rl}
    \textsc{Âge :}
      & \textbf{33} ans \\
    \textsc{Adresse :}
      & 29 Cours André Philip 69100 Villeurbanne \\
    \textsc{Téléphone :}
      & +33 6 89 36 32 19 \\
    \textsc{Email :}
      & \href{mailto:mounier.florian@gmail.com}{mounier.florian@gmail.com} \\
    \textsc{Site internet :}
      & \href{http://paradoxxxzero.github.com}{paradoxxxzero.github.io} \\
  \end{tabular} \\

  \section{Profil}
  \begin{tabular}{rl}
    & \textbf {Développeur full stack Python / ECMAScript} \\
    & \textbf{10 ans} d’expérience en conception réalisation de projets informatique \\
    & \textbf{8 ans} en Python / ECMAScript, \textbf{4 ans} en React / Redux \\
    & Ingénieur diplômé de l’INSA de Lyon,
    département informatique avec félicitations du jury \\
  \end{tabular} \\

  \section{Compétences techniques}
  \begin{tabular}{rl}
    \textsc{Langages :}
      & Python, ECMAScript, HTML5, CSS3, Sass, SQL \\
    \textsc{Bibliothèques :}
      & React, Redux, Flask, Jinja2, SQLAlchemy, lxml, d3.js \\
    \textsc{Serveurs :}
      & NGINX, uWSGI, Koa \\
    \textsc{Outils :}
      & Git, Github, Atom, Shell, Webpack, Pipenv, Yarn, Make, GitLab CI \\
    \textsc{Base de données :}
      & PostgreSQL \\
    \textsc{Systèmes :}
      & Linux (ArchLinux, Ubuntu/Debian), Windows, Android \\
  \end{tabular} \\

  \section{Formation et Langues}
  \begin{tabular}{rl}
    2009 − 2006 :       & INSA de Lyon second cycle département informatique \\
    2006 − 2004 :       & INSA de Lyon premier cycle \\
    \textsc{Anglais :}  & Courant \\
  \end{tabular} \\

  \section{Expériences professionnelles}
  \begin{supertabular}{r|p{11cm}}
    \multicolumn{2}{c}{} \\
    \textsc{Novembre} 2019  & \textbf{pour \textsc{Kozea} à Lyon} \footnotesize{9 ans} \\
    & \emph{Architecte Python ECMAScript / Responsable pôle innovation R\&D} \\
    & \footnotesize{
      Uniformisation de l’architecture des projets de l’entreprise :
      installation, pile technique, intégration continue et déploiement
    } \\
    & \footnotesize{
      Choix techniques pour la refonte complète de l’usine à site pharminfo.fr
      selon une architecture d’application monopage :
      \href{https://optiweb.pharminfo.fr}{optiweb.pharminfo.fr}
    } \\
    & \footnotesize{
      Transformation de la partie serveur Python en une API REST JSON
    } \\
    & \footnotesize{
      Réalisation de la partie client en React / Redux avec rendu côté serveur
      sous Webpack et Koa
    } \\
    & \footnotesize{
      Aide à la conception de l’interface dans une démarche orientée expérience
      utilisateur
    } \\
    & \footnotesize{
      Participation à la mise en place d’un fonctionnement en Holacratie
    } \\
    \\
    \footnotesize{\textsc{Décembre} 2015} & \emph{Ingénieur étude et développement} \\
    & \footnotesize{
      Participation à la réalisation et au maintient de la plateforme
      pharminfo.fr d’hébergement de sites internet de pharmacie :
      \href{https://www.pharminfo.fr}{pharminfo.fr}
    } \\
     \footnotesize{\textsc{Mars} 2011}
    & \begin{minipage}[b]{0.85\textwidth}
      \footnotesize{
        Réalisation d’une plateforme de groupement de pharmaciens,
        d’un site de formation en e-learning,
        d’une application Android de réservation d’ordonnance et
        d’une plateforme de gestion de prise de rendez-vous médecin
      }
    \end{minipage} \\

    \multicolumn{2}{c}{} \\

    \textsc{Février} 2011 & \textbf{pour \textsc{Objet Direct} à Lyon} \footnotesize{1 an et demi} \\
      & \emph{Ingénieur étude et développement} \\
      & \footnotesize{Accompagnement de divers clients dans l’univers Java J2EE} \\
      \footnotesize{\textsc{Mai} 2010} & \footnotesize{
        Mission en régie pour \textsc{Cegelec} à Saint Maurice De Beynost
      } \\
      \iflong
        & \emph{Étude et développement Java J2EE, Hibernate et JSF Icefaces Seam} \\
        & \footnotesize{
          Réalisation du Système d’Aide à la Gestion de Trafic de Vauban.
          Participation à la mise en place de l’architecture Seam.
          Réalisation des écrans de l’application en JSF Icefaces.
        } \\
        & \emph{Java J2EE, JBoss, Hibernate, Seam, JSF, Icefaces, Svn, Ant, PostgreSql} \\
      \fi
    \footnotesize{\textsc{Février} 2010} & \footnotesize{
      Mission en régie pour \textsc{Foederis} à Limonest
    } \\
      \iflong
        & \emph{Étude et développement Java J2EE, Hibernate et Struts} \\
        & \footnotesize{
          Participation à la réalisation du progiciel intranet de gestion de
          ressources humaines édité par Foederis.
          Maintenance de l’application, correction d’anomalies, refactoring de code.
          Réalisation d’évolutions, IHM / métier / persistance.
          Rédaction de documentation utilisateur, de documents d’analyse
          fonctionnelle et de documentation technique.
        } \\
        & \emph{Java J2EE, Tomcat, Hibernate, Struts, Selenium, Svn, Ant, Oracle, SQL Server} \\
        \footnotesize{\textsc{Décembre} 2009} & \footnotesize{Mission en régie pour \textsc{Cegelec} à Saint Maurice De Beynost} \\
        & \emph{Étude et développement Java J2EE, Jasper Reports} \\
        & \footnotesize{
          Participation au développement de l’application.
          Développement de rapports sous Jasper Reports et intégration de
          ces rapports dans un environnement web sous Struts2.
          Réorganisation des dépendances de librairies pour l’environnement de
          développement (Ant, Svn).
        } \\
        & \emph{Java J2EE, JBoss, Hibernate, Struts2, Jasper Reports, Svn, Ant, PostgreSql} \\
      \fi
    \footnotesize{\textsc{Septembre} 2009} & \footnotesize{
      Mission au forfait pour le \textsc{Brgm} à Lyon
    } \\
      \iflong
        & \emph{Étude et développement Java J2EE, Struts, XSL/XSLT} \\
        & \footnotesize{
          Participation aux développements de nouvelles fonctionnalités de
          l’application Géocatalogue
          (\href{http://www.geocatalogue.fr}{geocatalogue.fr})
          backoffice du Géoportail.
          Refactoring de composants.
          Intégration de la nouvelle charte graphique du client.
          Correction d’anomalies.
        } \\
        & \emph{Java J2EE, Tomcat, Hibernate, Struts, Spring, XSL/XSLT, CSS, Velocity, Exalead, Svn, Ant, PostgreSql} \\
      \fi

    \multicolumn{2}{c}{} \\

    \textsc{Janvier} 2009 & \textbf{pour \textsc{Objet Direct} à Lyon} \footnotesize{6 mois} \\
      & \emph{Projet de fin d’étude de conception et développement Java, GWT} \\
      \iflong
        & \footnotesize{
          Conception et développement d’un modeleur de diagramme UML open source
          en client léger Java, GWT.
          (Disponible à l’adresse
          \href{http://code.google.com/p/gwtuml/}{code.google.com/p/gwtuml}).
        } \\
        & \emph{Java, GWT, JavaScript, UML, Svn, Ant, Google code} \\
      \fi

    \multicolumn{2}{c}{} \\

    \textsc{Mai} 2008 & \textbf{pour l’\textsc{Atih} à Lyon} \footnotesize{4 mois} \\
      & \emph{Stage de conception et développement C\# .Net} \\
      \iflong
        & \footnotesize{
          Conception et développement de l’application DOMEVIH
          (DOssier MEdical du Virus de l’Immunodéficience Humaine)
          en client lourd C\# / .Net 2.0 pour l’agence ATIH
          (Agence Technique de l’Information sur l’Hospitalisation).
          Prestation réalisée dans le cadre d’un stage à temps plein
          au sein d’Euriware.
        } \\
        & \emph{C\#, .Net, Visual Source Safe, PostgreSql} \\
      \fi

    \multicolumn{2}{c}{} \\

    \textsc{Juin} 2007 & \textbf{pour la \textsc{Filière Ingénieur-Entreprendre} au Campus de la Doua} \footnotesize{2 mois} \\
      & \emph{Stage de développement PHP5} \\
      \iflong
        & \footnotesize{
          Développement d’une application web en PHP5 (objet) comprenant
          la mise à disposition d’un annuaire d’artisans interactif avec
          rating, blog, forum et services.
        } \\
        & \emph{PHP5, MySQL, HTML, CSS, CVS} \\
      \fi
  \end{supertabular} \\

  \section{Réalisations}
  \begin{tabular}{r|p{11cm}}
    \multicolumn{2}{c}{} \\
    \textsc{Formol}
      & Bibliothèque React d’édition de données arborescentes basée sur les formulaires HTML5 \\
      & \href{https://kozea.github.io/formol/}{kozea.github.io/formol} \\

    \multicolumn{2}{c}{} \\
    \textsc{Unrest}
      & Générateur d’API REST JSON extensibles à partir de modèles de données SQLAlchemy \\
      & \href{https://kozea.github.io/unrest/}{kozea.github.io/unrest} \\

    \multicolumn{2}{c}{} \\
    \textsc{Butterfly}
      & Implémentation web d’un terminal Linux basé sur WebSocket \\
      & \href{https://github.com/paradoxxxzero/butterfly}{github.com/paradoxxxzero/butterfly} \\

    \multicolumn{2}{c}{} \\
    \textsc{Pygal}
      & Bibliothèque Python de génération de graphiques vectoriels \\
      & \href{http://www.pygal.org/}{www.pygal.org} \\

    \multicolumn{2}{c}{} \\
    \textsc{Wdb}
      & Plateforme de débogage web client/serveur pour Python \\
      & \href{https://github.com/Kozea/wdb}{github.com/Kozea/wdb} \\

    \multicolumn{2}{c}{} \\
    \textsc{Umlaut}
      & Modeleur de diagramme en ligne basé sur d3.js \\
      & \href{http://kozea.github.io/umlaut/}{kozea.github.io/umlaut/} \\

    \multicolumn{2}{c}{} \\
    \textsc{Multicorn}
      & Bibliothèque C permettant l’inclusion de points d’accès python dans la définition de tables étrangères PostgreSQL \\
      & \href{https://multicorn.org/}{multicorn.org} \\

    \multicolumn{2}{c}{} \\
    \textsc{System monitor}
      & Extension de suivi des ressources système pour Gnome Shell \\
      & \href{https://github.com/paradoxxxzero/gnome-shell-system-monitor-applet}{github.com/paradoxxxzero/gnome-shell-system-monitor-applet} \\

    \multicolumn{2}{c}{} \\
    \textsc{Anakata}
      & Outil de visualisation d’objets géométriques en dimension 4 développé en ECMAScript, HTML5 Canvas et NoSql \\
      & \href{http://paradoxxxzero.github.io/anakata/}{paradoxxxzero.github.io/anakata} \\
  \end{tabular}
\end{document}
