\documentclass[a4paper,10pt]{article}
\RequirePackage{color,graphicx}
\usepackage{marvosym}
\usepackage{fontspec}
\usepackage{xunicode,xltxtra,url,parskip}
\usepackage[usenames,dvipsnames]{xcolor}
\usepackage[big]{layaureo}
\usepackage{supertabular}
\usepackage{titlesec}
\usepackage{hyperref}

\defaultfontfeatures{Mapping=tex-text}
\setmainfont[SmallCapsFont = Fontin SmallCaps]{Fontin}
\fontspec[Numbers={OldStyle}]{Fontin}

\definecolor{linkcolour}{rgb}{0.5, 0.5, 0.5}
\hypersetup{pdfborderstyle={/S/U/W 0.5}, urlbordercolor=linkcolour, breaklinks, colorlinks=false}

\titleformat{\section}{\Large\scshape\raggedright}{}{0em}{}[\titlerule]
\titlespacing{\section}{0pt}{3pt}{3pt}

\begin{document}
\pagestyle{empty}
\par{\centering
  {\Huge Florian \textsc{Mounier}
  }\bigskip\par}

\section{Informations personnelles}
\begin{tabular}{rl}
  \textsc{Âge :}        & \textbf{24} ans                                                    \\
  \textsc{Adresse :}    & 26 rue du Boeuf 69005 Lyon                                         \\
  \textsc{Téléphone :}  & +33 6 89 36 32 19                                                  \\
  \textsc{Email :}      & \href{mailto:mounier.florian@gmail.com}{mounier.florian@gmail.com} \\
\end{tabular}
\\
\section{Profil}
\begin{tabular}{rl}
  & \textbf {Ingénieur conception et développement}                                          \\
  & \textbf{2 ans} d'expérience en conception et développement objet à Objet Direct (Viseo)  \\
  & Ingénieur diplômé de l'INSA de Lyon, Département Informatique avec félicitations du jury \\
\end{tabular}
\\
\section{Compétences techniques}
\begin{tabular}{rl}
  \textsc{Langages :}        & Java J2EE, JavaScript, HTML5, CSS3, XML, XSL/XSLT, C\# .Net, C++, C, SQL, UML \\
  \textsc{Frameworks :}      & EJB3, Seam, JSF, Icefaces, Struts, Hibernate, jQuery, Dojo, GWT, Velocity     \\
  \textsc{Serveurs :}        & JBoss, Apache, Tomcat                                                         \\
  \textsc{Outils :}          & Svn, Git, Ant, Bash/Zsh, Eclipse, Emacs, Mantis / Trac, Jasper Reports        \\
  \textsc{Méthodes :}        & Design Patterns, UML                                                          \\
  \textsc{SGBDs :}           & PostgreSql, NoSql (MongoDB), MySql                                            \\
  \textsc{Systèmes :}        & Linux (ArchLinux, Ubuntu/Debian), Windows (7, Vista, XP, 98)                  \\
\end{tabular}
\\
\section{Formation et Langues}
\begin{tabular}{rl}
  \textsc{2009 - 2006}  & INSA de Lyon Second Cycle département Informatique.         \\
  \textsc{2006 - 2004}  & INSA de Lyon Premier Cycle.                                 \\
  \textsc{Anglais :}    & Lu, parlé, écrit.                                           \\
  \textsc{Allemand :}   & Lu, écrit                                                   \\
  \textsc{Italien :}    & Notions                                                     \\
\end{tabular}
\\
\section{Expériences professionnelles}
\begin{supertabular}{r|p{11cm}}
  \emph{Depuis} \textsc{Mai 2010}  & \textbf{pour \textsc{Cegelec} à Saint Maurice De Beynost}                           \\ 
  & \emph{Conception et développement Java J2EE, Hibernate et JSF Icefaces Seam}                                         \\
  & \footnotesize{Réalisation du Système d'Aide à la Gestion de Trafic de Vauban. Participation à la mise en place de l'architecture Seam. Réalisation des écrans de l'application en JSF Icefaces.} \\
  & \emph{Java J2EE, JBoss, Hibernate, Seam, JSF, Icefaces, Svn, Ant, PostgreSql}                                        \\
  \multicolumn{2}{c}{}                                                                                                   \\
  \textsc{Mai 2010} & \textbf{pour \textsc{Foederis} à Limonest}                                                         \\ 
  -- & \emph{Conception et développement Java J2EE, Hibernate et Struts}                                                 \\
  \footnotesize{\textsc{Février 2010}} & \footnotesize{Participation à la réalisation du progiciel intranet de gestion de ressources humaines édité par Foederis. Maintenance de l'application, correction d'anomalies, refactoring de code. Réalisation d'évolutions, IHM / métier / persistance. Rédaction de documentation utilisateur, de documents d'analyse fonctionnelle et de documentation technique.} \\
  & \emph{Java J2EE, Tomcat, Hibernate, Struts, Selenium, Svn, Ant, Oracle, SQL Server}                                  \\
  \multicolumn{2}{c}{}                                                                                                   \\
  \multicolumn{2}{c}{}                                                                                                   \\
  \textsc{Janvier 2010} & \textbf{pour \textsc{Cegelec} à Saint Maurice De Beynost}                                      \\ 
  -- & \emph{Conception et développement Java J2EE, Jasper Reports}                                                      \\
  \footnotesize{\textsc{Décembre 2009}}  & \footnotesize{Participation au développement de l'application. Développement de rapports sous Jasper Reports et intégration de ces rapports dans un environnement web sous Struts2. Réorganisation des dépendances de librairies pour l'environnement de développement (Ant, Svn).} \\
  & \emph{Java J2EE, JBoss, Hibernate, Struts2, Jasper Reports, Svn, Ant, PostgreSql}                                    \\
  \multicolumn{2}{c}{}                                                                                                   \\
  \textsc{Novembre 2009} & \textbf{pour le \textsc{Brgm} à Lyon}                                                         \\ 
  -- & \emph{Conception et développement Java J2EE, Struts, XSL/XSLT}                                                    \\
  \footnotesize{\textsc{Septembre 2009}} & \footnotesize{Participation aux développements de nouvelles fonctionnalités de l'application Géocatalogue (\href{http://www.geocatalogue.fr}{geocatalogue.fr}) backoffice du Géoportail. Refactoring de composants. Intégration de la nouvelle charte graphique du client. Correction d'anomalies.} \\
  & \emph{Java J2EE, Tomcat, Hibernate, Struts, Spring, XSL/XSLT, CSS, Velocity, Exalead, Svn, Ant, PostgreSql}          \\
  \multicolumn{2}{c}{}                                                                                                   \\
  \textsc{Juillet 2009} & \textbf{pour \textsc{Objet Direct} à Lyon}                                                     \\
  -- & \emph{Conception et développement Java, GWT}                                                                      \\
  \footnotesize{\textsc{Décembre 2008}}& \footnotesize{Conception et développement d’un modeleur de diagramme UML open source en client léger Java, GWT.  (Disponible à l'adresse \href{http://code.google.com/p/gwtuml/}{code.google.com/p/gwtuml}).} \\
  & \emph{Java, GWT, JavaScript, UML, Svn, Ant, Google code}                                                             \\
  \multicolumn{2}{c}{}                                                                                                   \\
  \textsc{Septembre 2008} & \textbf{pour l'\textsc{Atih} à Lyon}                                                         \\
  -- & \emph{Conception et développement C\# .Net}                                                                       \\
  \footnotesize{\textsc{Mai 2008}} & \footnotesize{Conception et développement de l’application DOMEVIH (DOssier MEdical du Virus de l’Immunodéficience Humaine) en client lourd C\# / .Net 2.0 pour l’agence ATIH (Agence Technique de l'Information sur l'Hospitalisation). Prestation réalisée dans le cadre d'un stage à temps plein au sein d'Euriware.} \\
  & \emph{C\#, .Net, Visual Source Safe, PostgreSql}                                                                     \\
  \multicolumn{2}{c}{}                                                                                                   \\
  \textsc{Août 2007} & \textbf{pour la \textsc{Filière Ingénieur-Entreprendre} au Campus de la Doua (Lyon)}              \\
  -- & \emph{Développement PHP5}                                                                                         \\
  \footnotesize{\textsc{Juin 2007}} & \footnotesize{Développement d’une application web en PHP5 (objet) comprenant la mise à disposition d’un annuaire d’artisans interactif avec rating, blog, forum et services.} \\
  & \emph{PHP5, MySQL, HTML, CSS, CVS}                                                                                   \\
  \multicolumn{2}{c}{}                                                                                                   \\
\end{supertabular}
\\
\section{Projets personnels}
\begin{tabular}{r|p{11cm}}
  \textsc{Anakata :}             & Outil de visualisation d'objets géométriques en dimension 4 développé en JavaScript et HTML5 canvas. Possibilité de sauvegarder les objets créés en local sur une base de donnée NoSql. \href{http://anakata.tk}{anakata.tk} \\
  \multicolumn{2}{c}{} \\
  \textsc{Imgur the world :}     & Extension open source pour le navigateur Google Chrome permettant d'envoyer directement des images sur l'hébergeur d'image \href{http://imgur.com}{imgur.com} (~2500 utilisateurs)                                            \\
  \multicolumn{2}{c}{} \\     
  \textsc{Reddit HD/Platinum :}  & Extensions open source pour le navigateur Google Chrome visant à ajouter une fonctionnalité de navigation au clavier au site d'actualité \href{http://www.reddit.com}{reddit.com} ainsi que d'en améliorer sa lisibilité.     \\
  \multicolumn{2}{c}{} \\   
\end{tabular}

\section{Intérêts}
Logiciels libres / Open Source \\
Programmation \\
Jeux de sociétés
\end{document}
