\documentclass[a4paper,10pt]{article}

\usepackage[sfdefault,oldstyle]{roboto}
\usepackage[utf8x]{inputenc}
\usepackage[T1]{fontenc}
\usepackage[english]{babel}

\usepackage[big]{layaureo}
\usepackage{supertabular}
\usepackage{titlesec}
\usepackage{hyperref}
\usepackage{xcolor}

\definecolor{softBlack}{cmyk}{0, 0, 0, 0.85}
\color{softBlack}

\hypersetup{urlcolor=blue, colorlinks=true}

\titleformat{\section}{\Large\scshape\raggedright}{}{0em}{}[\titlerule]
\titlespacing{\section}{0pt}{3pt}{3pt}

\begin{document}
\pagestyle{empty}
\par{\centering
  {\Huge Florian \textsc{Mounier}
  }\bigskip\par}

\section{Informations personnelles}
\begin{tabular}{rl}
  \textsc{Âge :}           & \textbf{33} ans \\
  \textsc{Adresse :}       & 29 Cours André Philip 69100 Villeurbanne \\
  \textsc{Téléphone :}     & +33 6 89 36 32 19 \\
  \textsc{Email :}         & \href{mailto:mounier.florian@gmail.com}{mounier.florian@gmail.com} \\
  \textsc{Site internet :} & \href{http://paradoxxxzero.github.com}{paradoxxxzero.github.io} \\
\end{tabular} \\

\section{Profil}
\begin{tabular}{rl}
  & \textbf {Ingénieur conception et développement} \\
  & \textbf{10 ans} d’expérience en conception réalisation de projets informatique. \\
  & Ingénieur diplômé de l’INSA de Lyon, Département Informatique avec félicitations du jury. \\
\end{tabular} \\

\section{Compétences techniques}
\begin{tabular}{rl}
  \textsc{Langages :}        & Python, JavaScript, HTML5, CSS3, Sass, SQL \\
  \textsc{Bibliothèques :}   & React, Redux, Flask, Jinja2, SQLAlchemy, lxml, d3.js \\
  \textsc{Serveurs :}        & nginx, uwsgi \\
  \textsc{Outils :}          & Git, Github, Emacs, Atom, Shell, Webpack, pipenv, npm, yarn, Make, Gitlab CI \\
  \textsc{Base de données :} & PostgreSQL \\
  \textsc{Systèmes :}        & Linux (ArchLinux, Ubuntu/Debian), Windows, Android \\
\end{tabular} \\

\section{Formation et Langues}
\begin{tabular}{rl}
  2009 − 2006  & INSA de Lyon Second Cycle département Informatique. \\
  2006 − 2004  & INSA de Lyon Premier Cycle. \\
  \textsc{Anglais :}    & Lu, parlé, écrit. \\
  \textsc{Allemand :}   & Lu, écrit. \\
  \textsc{Italien :}    & Notions. \\
\end{tabular} \\

\section{Expériences professionnelles}
\begin{supertabular}{r|p{11cm}}
  \multicolumn{2}{c}{} \\
  \emph{Depuis} \textsc{Mars} 2011  & \textbf{pour \textsc{Kozea} à Lyon} \\
  & \emph{Architecte Python JavaScript / Responsable pôle innovation R\&D} \\
  & \footnotesize{Participation à la réalisation et au maintient de la plateforme \textbf{pharminfo} d’hébergement de sites internet de pharmacie : \href{https://www.pharminfo.fr}{pharminfo.fr}} \\
  & \footnotesize{Choix techniques pour la refonte de cette usine à sites selon une architecture client/serveur avec une interface moderne : \href{https://optiweb.pharminfo.fr}{optiweb.pharminfo.fr}} \\
  & \footnotesize{Réécriture de la partie serveur Python devenue API à l’aide de la bibliothèque \href{https://kozea.github.io/unrest/}{kozea.github.io/unrest} créée pour l’occasion} \\
  & \footnotesize{Réalisation de la partie client React/Redux avec rendu serveur sous Webpack avec développement de bibliothèque \href{https://kozea.github.io/formol/}{kozea.github.io/formol} de génération de formulaire} \\
  & \footnotesize{Uniformisation de l’architecture des projets de l’entreprise : arborescence, pile technique et script Make unifié} \\
  & \footnotesize{Architecture et développement de la bibliothèque python \textbf{pygal} de génération de graphiques vectoriels : \href{http://www.pygal.org/en/stable/}{pygal.org}} \\
  & \footnotesize{Architecture et développement de la plateforme de débogage web client/serveur pour python \textbf{wdb} : \href{https://github.com/Kozea/wdb}{github.com/Kozea/wdb}} \\
  & \footnotesize{Membre de l’équipe de développement du projet \textbf{Multicorn} bibliothèque C permettant l’inclusion de points d'accès python dans la définition de tables étrangères PostgreSQL : \href{https://multicorn.org/}{multicorn.org}} \\


  \multicolumn{2}{c}{} \\
  \textsc{Mai} 2010  & \textbf{pour \textsc{Cegelec} à Saint Maurice De Beynost} \\
  -- & \emph{Conception et développement Java J2EE, Hibernate et JSF Icefaces Seam} \\
  \footnotesize{\textsc{Février} 2011} & \footnotesize{Réalisation du Système d’Aide à la Gestion de Trafic de Vauban. Participation à la mise en place de l’architecture Seam. Réalisation des écrans de l’application en JSF Icefaces.} \\
  & \emph{Java J2EE, JBoss, Hibernate, Seam, JSF, Icefaces, Svn, Ant, PostgreSql} \\


  \multicolumn{2}{c}{} \\
  \textsc{Mai} 2010 & \textbf{pour \textsc{Foederis} à Limonest} \\
  -- & \emph{Conception et développement Java J2EE, Hibernate et Struts} \\
  \footnotesize{\textsc{Février 2010}} & \footnotesize{Participation à la réalisation du progiciel intranet de gestion de ressources humaines édité par Foederis. Maintenance de l’application, correction d’anomalies, refactoring de code. Réalisation d’évolutions, IHM / métier / persistance. Rédaction de documentation utilisateur, de documents d’analyse fonctionnelle et de documentation technique.} \\
  & \emph{Java J2EE, Tomcat, Hibernate, Struts, Selenium, Svn, Ant, Oracle, SQL Server} \\


  \multicolumn{2}{c}{} \\
  \textsc{Janvier} 2010 & \textbf{pour \textsc{Cegelec} à Saint Maurice De Beynost} \\
  -- & \emph{Conception et développement Java J2EE, Jasper Reports} \\
  \footnotesize{\textsc{Décembre 2009}}  & \footnotesize{Participation au développement de l’application. Développement de rapports sous Jasper Reports et intégration de ces rapports dans un environnement web sous Struts2. Réorganisation des dépendances de librairies pour l’environnement de développement (Ant, Svn).} \\
  & \emph{Java J2EE, JBoss, Hibernate, Struts2, Jasper Reports, Svn, Ant, PostgreSql} \\


  \multicolumn{2}{c}{} \\
  \textsc{Novembre} 2009 & \textbf{pour le \textsc{Brgm} à Lyon} \\
  -- & \emph{Conception et développement Java J2EE, Struts, XSL/XSLT} \\
  \footnotesize{\textsc{Septembre 2009}} & \footnotesize{Participation aux développements de nouvelles fonctionnalités de l’application Géocatalogue (\href{http://www.geocatalogue.fr}{geocatalogue.fr}) backoffice du Géoportail. Refactoring de composants. Intégration de la nouvelle charte graphique du client. Correction d’anomalies.} \\
  & \emph{Java J2EE, Tomcat, Hibernate, Struts, Spring, XSL/XSLT, CSS, Velocity, Exalead, Svn, Ant, PostgreSql} \\


  \multicolumn{2}{c}{} \\
  \textsc{Juillet} 2009 & \textbf{pour \textsc{Objet Direct} à Lyon} \\
  -- & \emph{Conception et développement Java, GWT} \\
  \footnotesize{\textsc{Décembre} 2008}& \footnotesize{Conception et développement d’un modeleur de diagramme UML open source en client léger Java, GWT. (Disponible à l’adresse \href{http://code.google.com/p/gwtuml/}{code.google.com/p/gwtuml}).} \\
  & \emph{Java, GWT, JavaScript, UML, Svn, Ant, Google code} \\


  \multicolumn{2}{c}{} \\
  \textsc{Septembre} 2008 & \textbf{pour l’\textsc{Atih} à Lyon} \\
  -- & \emph{Conception et développement C\# .Net} \\
  \footnotesize{\textsc{Mai} 2008} & \footnotesize{Conception et développement de l’application DOMEVIH (DOssier MEdical du Virus de l’Immunodéficience Humaine) en client lourd C\# / .Net 2.0 pour l’agence ATIH (Agence Technique de l’Information sur l’Hospitalisation). Prestation réalisée dans le cadre d’un stage à temps plein au sein d’Euriware.} \\
  & \emph{C\#, .Net, Visual Source Safe, PostgreSql} \\


  \multicolumn{2}{c}{} \\
  \textsc{Août} 2007 & \textbf{pour la \textsc{Filière Ingénieur-Entreprendre} au Campus de la Doua (Lyon)} \\
  -- & \emph{Développement PHP5} \\
  \footnotesize{\textsc{Juin} 2007} & \footnotesize{Développement d’une application web en PHP5 (objet) comprenant la mise à disposition d’un annuaire d’artisans interactif avec rating, blog, forum et services.} \\
  & \emph{PHP5, MySQL, HTML, CSS, CVS} \\

\end{supertabular} \\

\section{Réalisations}

\begin{tabular}{r|p{11cm}}
  \multicolumn{2}{c}{} \\
  \textsc{Butterfly :} &  Implémentation web d'un terminal linux basé sur WebSocket. \href{https://github.com/paradoxxxzero/butterfly}{github.com/paradoxxxzero/butterfly}\\

  \multicolumn{2}{c}{} \\
  \textsc{Umlaut :} & Modeleur de diagramme en ligne basé sur d3.js. \href{http://kozea.github.io/umlaut/}{kozea.github.io/umlaut/} \\

  \multicolumn{2}{c}{} \\
  \textsc{System monitor :} & Extension de suivi des ressources système pour Gnome Shell. \href{https://github.com/paradoxxxzero/gnome-shell-system-monitor-applet}{github.com/paradoxxxzero/gnome-shell-system-monitor-applet} \\

  \multicolumn{2}{c}{} \\
  \textsc{Anakata :} & Outil de visualisation d’objets géométriques en dimension 4 développé en JavaScript, HTML5 canvas et NoSql \href{http://paradoxxxzero.github.io/anakata/}{paradoxxxzero.github.io/anakata} \\

\end{tabular}

\section{Intérêts}
Logiciels libres / Open Source \\
Programmation
\end{document}
